\documentclass[]{book}
\usepackage{lmodern}
\usepackage{amssymb,amsmath}
\usepackage{ifxetex,ifluatex}
\usepackage{fixltx2e} % provides \textsubscript
\ifnum 0\ifxetex 1\fi\ifluatex 1\fi=0 % if pdftex
  \usepackage[T1]{fontenc}
  \usepackage[utf8]{inputenc}
\else % if luatex or xelatex
  \ifxetex
    \usepackage{mathspec}
  \else
    \usepackage{fontspec}
  \fi
  \defaultfontfeatures{Ligatures=TeX,Scale=MatchLowercase}
\fi
% use upquote if available, for straight quotes in verbatim environments
\IfFileExists{upquote.sty}{\usepackage{upquote}}{}
% use microtype if available
\IfFileExists{microtype.sty}{%
\usepackage{microtype}
\UseMicrotypeSet[protrusion]{basicmath} % disable protrusion for tt fonts
}{}
\usepackage[unicode=true]{hyperref}
\hypersetup{
            pdfborder={0 0 0},
            breaklinks=true}
\urlstyle{same}  % don't use monospace font for urls
\usepackage{longtable,booktabs}
\usepackage{graphicx,grffile}
\makeatletter
\def\maxwidth{\ifdim\Gin@nat@width>\linewidth\linewidth\else\Gin@nat@width\fi}
\def\maxheight{\ifdim\Gin@nat@height>\textheight\textheight\else\Gin@nat@height\fi}
\makeatother
% Scale images if necessary, so that they will not overflow the page
% margins by default, and it is still possible to overwrite the defaults
% using explicit options in \includegraphics[width, height, ...]{}
\setkeys{Gin}{width=\maxwidth,height=\maxheight,keepaspectratio}
\IfFileExists{parskip.sty}{%
\usepackage{parskip}
}{% else
\setlength{\parindent}{0pt}
\setlength{\parskip}{6pt plus 2pt minus 1pt}
}
\setlength{\emergencystretch}{3em}  % prevent overfull lines
\providecommand{\tightlist}{%
  \setlength{\itemsep}{0pt}\setlength{\parskip}{0pt}}
\setcounter{secnumdepth}{5}
% Redefines (sub)paragraphs to behave more like sections
\ifx\paragraph\undefined\else
\let\oldparagraph\paragraph
\renewcommand{\paragraph}[1]{\oldparagraph{#1}\mbox{}}
\fi
\ifx\subparagraph\undefined\else
\let\oldsubparagraph\subparagraph
\renewcommand{\subparagraph}[1]{\oldsubparagraph{#1}\mbox{}}
\fi

\author{}
\date{\vspace{-2.5em}}

\begin{document}

{
\setcounter{tocdepth}{1}
\tableofcontents
}
\chapter{Syllabus}\label{syllabus}

\section{Instructor Information}\label{instructor-information}

Instructor: Dr.~Amy Hurford\\
Office: Teaching remotely\\
Email: \href{mailto:ahurford@mun.ca}{\nolinkurl{ahurford@mun.ca}}\\
WebEx: \url{https://mun.webex.com/meet/ahurford}\\
Course website: \url{https://ahurford.github.io/math-4190/}

Availability: I will try to reply to emails within 24 hours (excluding
evenings, weekends and holidays). I am always available during the
lecture times. Please email to request a meeting for a different time.
Please check my \href{https://amyhurford.weebly.com/}{schedule} and
suggest a time I am free that works for you.

\section{Course Information}\label{course-information}

TR 2-3.15pm meet on WebEx

Course description:\\
MATH 4190 Mathematical Modelling is intended to develop students' skills
in mathematical modelling and competence in oral and written
presentations. Case studies in modelling will be analyzed. Students will
develop a mathematical model and present it in both oral and report
form.

Course format:\\
For the first 7 weeks of class, each week you will have an assignment to
complete. There will not be lectures, but there may be required readings
(ideally to be completed before class). For the next 5 weeks, during
class time you should work on your final project. During the last week
of class each student will do an oral presentation of their final
project. During classtime, I will be available on WebEx to help you with
your assignments, to answer your questions, or to advise you regarding
your final project. If you are not able to make it to class, but require
help, please email me to set up an appointment.

Course expectations:\\
Any students that are disruptive, violating university policies, or
acting in a potentially unsafe way will be warned and asked to
leave.\\[2\baselineskip]Learning goals:\\
This course will teach you how to derive, parameterize, and interpret
your own mathematical models with an emphasis on `hands-on' modelling
experience.

Required Text and Resources:\\
The ebook at \url{https://ahurford.github.io/math-4190/} is the text for
the course. This ebook will refer you to any other readings that will be
either publically available or available via the MUN library. Class
announcements and submission of your assignments will occur through
BrightSpace.

\section{Method of Evaluation}\label{method-of-evaluation}

\begin{itemize}
\tightlist
\item
  7 assignments (equal weighting) - 35\%
\item
  Oral presentation (week of April 5) - 15\%
\item
  Written project (due Monday April 12 at 9am) - 50\%
\end{itemize}

See Section \ref{final-project} for expectations regarding the final
project.

Late assignments, labs, and missed midterms, and final exams will be
accommodated as described by University Regulation 6.7.3 and 6.7.5 (see
\url{https://www.mun.ca/regoff/calendar/sectionNo=REGS-0474} for
Regulations).

\section{Additional Policies}\label{additional-policies}

\subsection{Accommodation of students with
disabilities}\label{accommodation-of-students-with-disabilities}

Memorial University of Newfoundland is committed to supporting inclusive
education based on the principles of equity, accessibility and
collaboration. Accommodations are provided within the scope of the
University Policies for the Accommodations for Students with
Disabilities see \url{www.mun.ca/policy/site/policy.php?id=239}.
Students who may need an academic accommodation are asked to initiate
the request with the Glenn Roy Blundon Centre at the earliest
opportunity (see \url{www.mun.ca/blundon} for more information).

\subsection{Academic misconduct}\label{academic-misconduct}

Students are expected to adhere to those principles, which constitute
proper academic conduct. A student has the responsibility to know which
actions, as described under Academic Offences in the University
Regulations, could be construed as dishonest or improper. Students found
guilty of an academic offence may be subject to a number of penalties
commensurate with the offence including reprimand, reduction of grade,
probation, suspension or expulsion from the University. For more
information regarding this policy, students should refer to University
Regulation 6.12.

\subsection{Equity and Diversity}\label{equity-and-diversity}

A safe learning environment will be provided for all students regardless
of race, colour, nationality, ethnic origin, social origin, religious
creed, religion, age, disability, disfigurement, sex (including
pregnancy), sexual orientation, gender identity, gender expression,
marital status, family status, source of income or political opinion.

You should not photograph or record myself, teaching assistants, or
other students in the class without first obtaining permission.
Accommodation will be made for students with special needs.

The sound should be turned off on phones and computers during class.

\section{Additional Supports}\label{additional-supports}

Resources for additional support can be found at:

\begin{itemize}
\item
  \url{www.mun.ca/currentstudents/student/}
\item
  \url{https://munsu.ca/resource-centres/}
\end{itemize}

\section{Tentative course schedule}\label{tentative-course-schedule}

The course schedule is found in the toolbar of the class materials, see
\url{https://ahurford.github.io/math-4190/}.

The last day to drop the course without academic prejudice is Monday
March 8.

\chapter{A1. Tues Jan 12: What is a mathematical
model?}\label{a1.-tues-jan-12-what-is-a-mathematical-model}

Assignment 1: to be handed in on Brightspace by Tues Jan 19 at 2pm.

\section*{Questions}\label{questions}
\addcontentsline{toc}{section}{Questions}

\begin{enumerate}
\def\labelenumi{\arabic{enumi}.}
\item
  What is a mathematical model? Give the citation for your answer in the
  style of the Bulletin of Mathematical Biology. This citation must be
  to a book (i.e., from the MUN library, or another an article from the
  peer-reviewed literature). {[}2 marks{]}
\item
  Usually statistical models such as a linear regression satisfy my
  defintion of a mathematical model. Given your definition in 1., would
  a linear regression meet the requirements for a mathematical model?
  {[}1 mark{]}
\item
  Per your answer to 1., give an example of something that is a model,
  but not a mathematical model. {[}1 mark{]}
\item
  List some types of equations that are commonly used as models. {[}1
  mark{]}
\end{enumerate}

While the definition of a mathematical model is quite broad, in keeping
with the pre-requisites for this class, we will emphasize models from
dynamical systems throughout this course.

You may search the MUN library online holdings to find a book or
peer-reviewed article to answer the above questions. Alternatively, you
may use one of the resources below:

Bliss et al. 2014
\href{https://m3challenge.siam.org/sites/default/files/uploads/siam-guidebook-final-press.pdf}{Math
modeling: getting started and getting solutions}

Otto and Day 2009.
\href{https://ebookcentral-proquest-com.qe2a-proxy.mun.ca/lib/MUN/detail.action?docID=768551}{A
biologists guide to mathematical modelling}

\href{https://link-springer-com.qe2a-proxy.mun.ca/book/10.1007\%2F978-3-030-44541-6}{A
primer on mathematical modelling} (the chapters can be downloaded for
free)

\chapter{A1. Thurs Jan 14: Getting
started}\label{a1.-thurs-jan-14-getting-started}

Assignment 1: to be handed in on Brightspace by Tues Jan 19 at 2pm.

When deriving a mathematical model, some practical advice is:

\begin{enumerate}
\def\labelenumi{\alph{enumi}.}
\tightlist
\item
  Modifying an existing model to suit your purposes is often a good
  approach; and\\
\item
  The units in the terms of your model need to be consistent.\\
\end{enumerate}

We would like to derive a mathematical model for the following:

\emph{Cooling beer in the freezer}\\
Room temperature beer can be cooled down by putting it in the freezer.
If left too long, the beer will freeze and burst the glass making it
undrinkable. How long can the beer be left in the freezer?

Read Box 2.1 on p30-31 of
\href{https://ebookcentral-proquest-com.qe2a-proxy.mun.ca/lib/MUN/detail.action?docID=768551}{A
biologists guide to mathematical modelling}.

Step 1 of the modelling process has been completed because the question
has been provided. Next, we will skip to Step 4. Read about the
\emph{cooling cup of coffee} model below. We can borrow the
\emph{cooling cup of coffee} model to use as our model for the
\emph{cooling beer in the freezer} question.

\section{The cooling cup of coffee}\label{the-cooling-cup-of-coffee}

The following is taken from Barnes \& Fulford (2002), section 9.1-9.2
starting of p218.

We would like to know how long it will take a 60\(^\circ\)C cup of
coffee to drop to 40\(^\circ\)C. To develop the model, we assume the cup
of coffee is a uniform temperature throughout. The cup of coffee will
cool as heat energy from the coffee is lost to its surroundings, which
are at a lower temperature. Note that temperature is measured in degrees
Celsius, while heat is measured in Joules or Watts (Joules per second).

From physics, the equation for the change in the heat content of coffee,
\(Q_{hc}\) (in Watts) is,

\begin{equation}
Q_{hc} = cm \frac{dU(t)}{dt},
\end{equation}

where \(c\) is the heat specific constant (measured in Joules per degree
Celcius per kilogram), \(m\) is the mass of the material being heated or
cooled (in kilograms), and \(U(t)\) is the temperature in Celcius.

Also, from physics the rate of heat exchange of an object which is
hotter than its surroundings, \(Q_{cs}\), (in Watts) is,

\begin{equation}
Q_{cs} = hS(U(t)-u_s),
\end{equation}

where \(h\) is the constant of convective heat transfer (in Watts per
\(m^2\) per degree Celcuis), \(S\) is the surface area from which heat
is lost, \(U(t)\) is the temperature of the object and \(u_s<U(t)\) is
the temperature of the surroundings (in Celcius).

Owing to the conservation of heat energy, the change in the heat content
of coffee, \(Q_{hc}\), should be equal to the rate of heat exchange from
the coffee to its surroundings, \(Q_{cs}\). As such, the equation for
the cooling of the cup of coffee is,

\begin{equation}
cm \frac{dU(t)}{dt}=-hS(U(t)-u_s).
\label{eq:LL}
\end{equation}

The negative sign on the righthand side of equation \eqref{eq:LL} is to
make sure both sides of the equation are consistent because
\(\frac{dU(t)}{dt}\) is negative while all other terms are positive.
This is a linear ordinary differential equation (ODE). It is possible to
integrate this ODE to solve for \(U(t)\). If we assume that
\(U(0) = 60^\circ C\) we can then solve \(t_40\) such that
\(U(t_40) = 40^\circ C\).

\section*{Questions}\label{questions}
\addcontentsline{toc}{section}{Questions}

\begin{enumerate}
\def\labelenumi{\arabic{enumi}.}
\setcounter{enumi}{4}
\item
  Complete Step 4 from Box 2.1 on p30-31 of
  \href{https://ebookcentral-proquest-com.qe2a-proxy.mun.ca/lib/MUN/detail.action?docID=768551}{A
  biologists guide to mathematical modelling} for the cooling beer
  problem. Write down the equation for the cooling beer model, define
  all the parameters and variables, also giving their units, contraints,
  and values. Calculate the units for each term in your equation to make
  sure the units are consistent across terms (i.e., perform a
  \href{https://en.wikipedia.org/wiki/Dimensional_analysis}{dimensional
  analysis}).\\[2\baselineskip]Note the difference between
  \emph{parameters} and \emph{variables}. For dynamical system models
  the dependent \emph{variables} are the quantities that change over
  time. Time is the independent variable. \emph{Parameters} are
  constants whose values are estimated. The dependent variables need to
  be assigned an initial value.\\[2\baselineskip]You will need to do
  some research to estimate the values of the parameters and the initial
  condition for the variable in your model. Provide your evidence to
  support your parameter estimates.\\
  List any assumptions that you have made, either in your model
  formulation or regarding the parameter estimates.
\item
  Complete Steps 5-7 from Box 2.1 on p30-31 of
  \href{https://ebookcentral-proquest-com.qe2a-proxy.mun.ca/lib/MUN/detail.action?docID=768551}{A
  biologists guide to mathematical modelling} for the cooling beer
  problem.
\end{enumerate}

\chapter{A2. Tues Jan 19: Deriving ordinary differential equation
models}\label{a2.-tues-jan-19-deriving-ordinary-differential-equation-models}

Assignment 2: to be handed in to Brightspace on Tues Jan 26 by 2pm.

Ordinary differential equations (ODEs) models are often appropriate for
MATH 4190 final projects. The general form of an ODE model is:

\begin{equation}\label{eq:genCT}
\frac{dn(t)}{dt} = \mbox{rate of increase} - \mbox{rate of decrease}.
\end{equation}

Using this general form, familiarizing yourself with some classic ODE
models, ensuring the units of the terms in your ODE are consistent, and
knowing the assumptions of ODE models,

\section{HIV model}\label{hiv-model}

This model is modified from Otto and Day (2009).

\emph{Variables}\\
\(S(t)\): number of susceptible T-cells (number, \(S(t)\geq 0\)).\\
\(L(t)\): number of latently infected T-cells (number,
\(L(t)\geq 0\)).\\
\(A(t)\): number of actively infected cells T-cells (number,
\(A(t)\geq 0\)).\\
\(V(t)\): amount of virus (number, \(V(t)\geq 0\)).\\

\emph{Parameters}\\
\(\Sigma:\) rate of production of susceptible T-cells (number/time,
\(\Sigma \geq 0\)).\\
\(d:\) removal rate of T-cells (1/time, \(d \geq 0\)).\\
\(\beta:\) rate of T-cell infection (1/number/time, \(\beta >0\)).\\
\(p:\) fraction of infected T-cells that upon infection are latently
infected (unitless, \(1\geq p \geq 0\)).\\
\(a:\) rate that latent T-cells become activated (1/time,
\(a \geq 0\)).\\
\(\delta:\) death rate/removal of actively infected T-cells (1/time,
\(\delta \geq d\)).\\
\(\pi:\) rate that virus is produced by actively infected T-cells
(1/time, \(\pi > 0\)).\\
\(\kappa:\) rate of virus removal (1/time, \(\kappa > 0\).)

to be continued\ldots{}

\chapter{A2. Thurs Jan 21: Deriving recursion equation
models}\label{a2.-thurs-jan-21-deriving-recursion-equation-models}

Assignment 2: to be handed in to Brightspace on Tues Jan 26 by 2pm.

\chapter{A3. Tues Jan 26: Analysis of dynamical
systems}\label{a3.-tues-jan-26-analysis-of-dynamical-systems}

Assignment 3: to be handed in to Brightspace on Tues Feb 2 by 2pm.

\chapter{A3. Thurs Jan 28: Classic models of population
biology}\label{a3.-thurs-jan-28-classic-models-of-population-biology}

Assignment 3: to be handed in to Brightspace on Tues Feb 2 by 2pm.

\chapter{A4. Tues Feb 2: Epidemic, consumer-resource, and competition
models}\label{a4.-tues-feb-2-epidemic-consumer-resource-and-competition-models}

Assignment 4: to be handed in to Brightspace on Tues Feb 9 by 2pm.

\chapter{A4. Thurs Feb 4: Partial differential and delay differential
equation
models}\label{a4.-thurs-feb-4-partial-differential-and-delay-differential-equation-models}

Assignment 4: to be handed in to Brightspace on Tues Feb 9 by 2pm.

\chapter{A5. Tues Feb 9: Parameter
estimation}\label{a5.-tues-feb-9-parameter-estimation}

Assignment 5: to be handed in to Brightspace on Tues Feb 16 by 2pm.

\chapter{A5. Thurs Feb 11: Numerical
solultions}\label{a5.-thurs-feb-11-numerical-solultions}

Assignment 5: to be handed in to Brightspace on Tues Feb 16 by 2pm.

\chapter{A6. Tues Feb 16: Case
studies}\label{a6.-tues-feb-16-case-studies}

Assignment 6: to be handed in to Brightspace on Tues Feb 27 by 2pm.

\chapter{A6. Thurs Feb 18: Make your own
model}\label{a6.-thurs-feb-18-make-your-own-model}

Assignment 6: to be handed in to Brightspace on Tues Feb 27 by 2pm.

\chapter{A7. Tues Feb 23: Stochastic
models}\label{a7.-tues-feb-23-stochastic-models}

Assignment 7: to be handed in to Brightspace on Tues Mar 2 by 2pm.

\chapter{A7. Thurs Feb 25: Simulation of stochastic
models}\label{a7.-thurs-feb-25-simulation-of-stochastic-models}

Assignment 7: to be handed in to Brightspace on Tues Mar 2 by 2pm.

\chapter{Final project}\label{final-project}

\begin{itemize}
\tightlist
\item
  Oral presentation - 15\% (week of April 5)
\item
  Written Project - 50\% (due Monday April 12 at 9am)
\end{itemize}

The main focus of the second half of MATH 4190 is the final project. You
are to:

\begin{itemize}
\tightlist
\item
  derive and analyze an original mathematical model,
\item
  extend and analyze an existing mathematical model, or
\item
  explain, recreate and/or produce novel proofs for an existing analysis
  of a mathematical model involving advanced mathematics (i.e., graduate
  level).
\end{itemize}

\section{Written Report}\label{written-report}

Your written report should consist of four sections: Introduction,
Model, Results, and Discussion (for more details see below). You must
appropriately acknowledge results and models that are not your own work.
Your written report must be no more than 10 pages (excluding references,
figures, and appendices) and written in Latex. If you have more than 10
pages of important content, then you should choose to place some of this
content in an Appendix.

Some advice: - Start simple. Start with a model that you are confident
that you will be able to produce some results for. When you fully
understand simpler versions of your model, gradually add in more
complexity. This prevents you from tackling a project this is `too hard'
and not getting any results. - Be concise. A good project is
thoughtfully put together -- it does not necessarily need to be overly
complex or long.

Guidelines for each section of the final project:

\subsection{Introduction}\label{introduction}

\begin{itemize}
\tightlist
\item
  Describe the general question/problem of interest.
\item
  Describe the real-world application in language that is accessible to
  a reader without prior knowledge of the application.
\item
  Summarize relevant previous research from the scientific literature
  (i.e., scholarly journals and books) and clearly place the project
  within the context of what has already been done. Remember to cite
  modeling articles, not just articles that describe the details of your
  applied system.
\item
  Don't focus your literature research to narrowly. For example, if you
  project is on `moose-wolf population dynamics' consider also the
  literature on predator-prey dynamics, which may contain mathematically
  equivalent model formulations.
\item
  Concisely state the main objective, hypothesis or question of the
  project.
\item
  Usually ends with a paragraph describing how the problem will be
  solved, i.e., the type of model and the type of analysis.
\end{itemize}

\subsection{Model}\label{model}

\begin{itemize}
\tightlist
\item
  Your model should be a dynamical system. The model cannot be an
  autonomous linear system of differential equations or difference
  equations because the analysis of such a model is too simple. You may
  however, consider a simplification of the model you are interested in
  that is an autonomous linear system of equations, if it helps you to
  understand the non-linear or non-autonomous system of interest.
\item
  Provide all details of the model to be analyzed. Define all the model
  parameters, variables, provide their word definitions and state their
  units. Provide the complete system of equations that comprises the
  model. If appropriate provide a table of parameter values (and
  references if literature sources are used to justify choices of
  parameter values).
\item
  You should describe the assumptions of your model.
\item
  You should provide enough detail that your model is reproducible. This
  may mean that you need to appendicize some material.
\item
  Consider including a diagram that communicates the interactions
  present in your model. This is often useful.
\end{itemize}

\subsection{Results}\label{results}

\begin{itemize}
\tightlist
\item
  If possible, consider special cases of your model that yield
  analytical solutions such as equilibria and local stability
  conditions.
\item
  If it is possible to solve for the equilibrium values and/or to
  determine the local stability of equilibria for your model you must do
  so.
\item
  Usually the results section of a MATH 4190 project includes numerical
  solutions for the main model of interest.
\item
  You may wish to summarize how numerical results for your model change
  depending on the parameter values used.
\item
  Depending on your project, you may wish to explore different types of
  more advanced analyses, i.e.~perturbation analysis, sensitivity and
  uncertainty analysis, stochastic simulations based on the Gillespie
  algorithm, bifurcation diagrams, analysis of periodic systems, and
  model validation.
\item
  You do not need to hand in your code. If your numerical methods are
  complex or non-standard you may wish to provide your pseudo-code in an
  Appendix.
\item
  Your results should include at least 1 figure that is fully labeled
  and contains a figure caption.
\item
  Your results section does not need to include every analysis that you
  have done. Your results section should be logically organized and this
  may mean omitting analyses that `didn't work out' or you judge to be
  less important given later results that you were able to achieve. Your
  results should be appropriate given the question/hypothesis that you
  described in the Introduction.
\end{itemize}

\subsection{Discussion}\label{discussion}

\begin{itemize}
\tightlist
\item
  Interpret your results in terms of the main question/hypothesis.
\item
  Towards the beginning of the Discussion, typically there is a section
  that reiterates the highlights of the results section.
\item
  Your discussion should describe whether your results matched what was
  expected or not. Are your results consistent with other similar
  studies?
\item
  You should interpret your results in the context of the applied
  problem, i.e., what does a stability condition suggest in terms of the
  practical management of a system?
\item
  How did your assumptions affect your results?
\item
  Do your results suggest any avenues for future research?
\item
  Highlight what is novel about your work.
\end{itemize}

You are to provide a complete bibliography of literature cited. You can
choose to use the referencing style of any scholarly scientific journal
or any of the preset options from Bibtex. The referencing style you use
must be consistent throughout.

\section{Oral report}\label{oral-report}

The content of your oral report should be similiar to the written
report.

I am here to help! Please let me know if you have any questions.

For writing tips, you may consider: (1)
\href{https://m3challenge.siam.org/sites/default/files/uploads/siam-guidebook-final-press.pdf}{Bliss
et al. 2014} (p42-44); (2)
\href{http://matryoshka.org/2012/07/13/how-to-write-a-paper-people-will-cite/}{``How
to write a theoretical ecology paper that people will cite''}; or (3)
\href{https://video.mbi.ohio-state.edu/video/player/?id=4775\&title=Webinar\%3A\%20How\%20to\%20Write\%20a\%20Modelling\%20Paper}{Webinar:
How to write a modelling paper}

\end{document}
