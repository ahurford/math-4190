\documentclass[]{book}
\usepackage{lmodern}
\usepackage{amssymb,amsmath}
\usepackage{ifxetex,ifluatex}
\usepackage{fixltx2e} % provides \textsubscript
\ifnum 0\ifxetex 1\fi\ifluatex 1\fi=0 % if pdftex
  \usepackage[T1]{fontenc}
  \usepackage[utf8]{inputenc}
\else % if luatex or xelatex
  \ifxetex
    \usepackage{mathspec}
  \else
    \usepackage{fontspec}
  \fi
  \defaultfontfeatures{Ligatures=TeX,Scale=MatchLowercase}
\fi
% use upquote if available, for straight quotes in verbatim environments
\IfFileExists{upquote.sty}{\usepackage{upquote}}{}
% use microtype if available
\IfFileExists{microtype.sty}{%
\usepackage{microtype}
\UseMicrotypeSet[protrusion]{basicmath} % disable protrusion for tt fonts
}{}
\usepackage[unicode=true]{hyperref}
\hypersetup{
            pdfborder={0 0 0},
            breaklinks=true}
\urlstyle{same}  % don't use monospace font for urls
\usepackage{longtable,booktabs}
\usepackage{graphicx,grffile}
\makeatletter
\def\maxwidth{\ifdim\Gin@nat@width>\linewidth\linewidth\else\Gin@nat@width\fi}
\def\maxheight{\ifdim\Gin@nat@height>\textheight\textheight\else\Gin@nat@height\fi}
\makeatother
% Scale images if necessary, so that they will not overflow the page
% margins by default, and it is still possible to overwrite the defaults
% using explicit options in \includegraphics[width, height, ...]{}
\setkeys{Gin}{width=\maxwidth,height=\maxheight,keepaspectratio}
\IfFileExists{parskip.sty}{%
\usepackage{parskip}
}{% else
\setlength{\parindent}{0pt}
\setlength{\parskip}{6pt plus 2pt minus 1pt}
}
\setlength{\emergencystretch}{3em}  % prevent overfull lines
\providecommand{\tightlist}{%
  \setlength{\itemsep}{0pt}\setlength{\parskip}{0pt}}
\setcounter{secnumdepth}{5}
% Redefines (sub)paragraphs to behave more like sections
\ifx\paragraph\undefined\else
\let\oldparagraph\paragraph
\renewcommand{\paragraph}[1]{\oldparagraph{#1}\mbox{}}
\fi
\ifx\subparagraph\undefined\else
\let\oldsubparagraph\subparagraph
\renewcommand{\subparagraph}[1]{\oldsubparagraph{#1}\mbox{}}
\fi

\author{}
\date{\vspace{-2.5em}}

\begin{document}

{
\setcounter{tocdepth}{1}
\tableofcontents
}
\chapter{Tues Sept 12: Syllabus}\label{tues-sept-12-syllabus}

\section{Instructor Information}\label{instructor-information}

Instructor: Dr.~Amy Hurford\\
Office: Teaching remotely\\
Email: \href{mailto:ahurford@mun.ca}{\nolinkurl{ahurford@mun.ca}}\\
WebEx: \url{https://mun.webex.com/meet/ahurford}\\
Course website: \url{https://ahurford.github.io/math-4190/}

Availability: I will try to reply to emails within 24 hours (excluding
evenings, weekends and holidays). I am always available during the
lecture times. Please email to request a meeting for a different time.
Please check my \href{https://amyhurford.weebly.com/}{schedule} and
suggest a time I am free that works for you.

\section{Course Information}\label{course-information}

TR 2-3.15pm meet on WebEx

Course description:\\
MATH 4190 Mathematical Modelling is intended to develop students' skills
in mathematical modelling and competence in oral and written
presentations. Case studies in modelling will be analyzed. Students will
develop a mathematical model and present it in both oral and report
form.

Course format:\\
For the first 7 weeks of class, each week you will have an assignment to
complete. There will not be lectures, but there may be required readings
(ideally to be completed before class). For the next 5 weeks, during
class time you should work on your final project. During the last week
of class each student will do an oral presentation of their final
project. During classtime, I will be available on WebEx to help you with
your assignments, to answer your questions, or to advise you regarding
your final project. If you are not able to make it to class, but require
help, please email me to set up an appointment.

Course expectations:\\
Any students that are disruptive, violating university policies, or
acting in a potentially unsafe way will be warned and asked to
leave.\\[2\baselineskip]Learning goals:\\
This course will teach you how to derive, parameterize, and interpret
your own mathematical models with an emphasis on `hands-on' modelling
experience.

Required Text and Resources:\\
The ebook at \url{https://ahurford.github.io/math-4190/} is the text for
the course. This ebook will refer you to any other readings that will be
either publically available or available via the MUN library. Class
announcements and submission of your assignments will occur through
BrightSpace.

\section{Method of Evaluation}\label{method-of-evaluation}

\begin{itemize}
\tightlist
\item
  6 assignments (equal weighting) - 35\%
\item
  Oral presentation - 15\% (week of April 5)
\item
  Final Project (due Monday April 12 at 9am) - 50\%
\end{itemize}

Late assignments, labs, and missed midterms, and final exams will be
accommodated as described by University Regulation 6.7.3 and 6.7.5 (see
\url{https://www.mun.ca/regoff/calendar/sectionNo=REGS-0474} for
Regulations).

\section{Additional Policies}\label{additional-policies}

\subsection{Accommodation of students with
disabilities}\label{accommodation-of-students-with-disabilities}

Memorial University of Newfoundland is committed to supporting inclusive
education based on the principles of equity, accessibility and
collaboration. Accommodations are provided within the scope of the
University Policies for the Accommodations for Students with
Disabilities see \url{www.mun.ca/policy/site/policy.php?id=239}.
Students who may need an academic accommodation are asked to initiate
the request with the Glenn Roy Blundon Centre at the earliest
opportunity (see \url{www.mun.ca/blundon} for more information).

\subsection{Academic misconduct}\label{academic-misconduct}

Students are expected to adhere to those principles, which constitute
proper academic conduct. A student has the responsibility to know which
actions, as described under Academic Offences in the University
Regulations, could be construed as dishonest or improper. Students found
guilty of an academic offence may be subject to a number of penalties
commensurate with the offence including reprimand, reduction of grade,
probation, suspension or expulsion from the University. For more
information regarding this policy, students should refer to University
Regulation 6.12.

\subsection{Equity and Diversity}\label{equity-and-diversity}

A safe learning environment will be provided for all students regardless
of race, colour, nationality, ethnic origin, social origin, religious
creed, religion, age, disability, disfigurement, sex (including
pregnancy), sexual orientation, gender identity, gender expression,
marital status, family status, source of income or political opinion.

You should not photograph or record myself, teaching assistants, or
other students in the class without first obtaining permission.
Accommodation will be made for students with special needs.

The sound should be turned off on phones and computers during class.

\section{Additional Supports}\label{additional-supports}

Resources for additional support can be found at:

\begin{itemize}
\item
  \url{www.mun.ca/currentstudents/student/}
\item
  \url{https://munsu.ca/resource-centres/}
\end{itemize}

\section{Tentative course schedule}\label{tentative-course-schedule}

The course schedule is found in the toolbar of the class materials, see
\url{https://ahurford.github.io/math-4190/}.

The last day to drop the course without academic prejudice is Monday
March 8.

\chapter{Quantitative skills laboratory}\label{lab2}

\begin{center}\rule{0.5\linewidth}{0.5pt}\end{center}

PURPOSE

\begin{enumerate}
\def\labelenumi{\arabic{enumi}.}
\tightlist
\item
  To learn how to record data in electronic format
\item
  To learn how to write hypotheses as equations
\item
  To learn how to choose the appropriate visualizations
\item
  To learn how to make graphs using R Studio
\end{enumerate}

\begin{center}\rule{0.5\linewidth}{0.5pt}\end{center}

Before coming to the laboratory:

\begin{enumerate}
\def\labelenumi{\arabic{enumi}.}
\item
  If you have not already, install
  \href{https://ahurford.github.io/quant-guide-all-courses/install.html}{\texttt{R}
  and \texttt{RStudio}}.
\item
  Before coming to the laboratory read:
\end{enumerate}

\begin{itemize}
\tightlist
\item
  \href{https://ahurford.github.io/quant-guide-all-courses/rintro.html}{Introduction
  to R}
\item
  \href{https://ahurford.github.io/quant-guide-all-courses/graph.html}{Making
  graphs in R}
\item
  \href{https://ahurford.github.io/quant-guide-all-courses/data-entry.html}{Entering
  and loading data}
\end{itemize}

\section*{EXERCISE 1. Data entry and graphing with a continuous
independent
variable}\label{exercise-1.-data-entry-and-graphing-with-a-continuous-independent-variable}
\addcontentsline{toc}{section}{EXERCISE 1. Data entry and graphing with
a continuous independent variable}

\end{document}
