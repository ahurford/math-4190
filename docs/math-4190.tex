\documentclass[]{book}
\usepackage{lmodern}
\usepackage{amssymb,amsmath}
\usepackage{ifxetex,ifluatex}
\usepackage{fixltx2e} % provides \textsubscript
\ifnum 0\ifxetex 1\fi\ifluatex 1\fi=0 % if pdftex
  \usepackage[T1]{fontenc}
  \usepackage[utf8]{inputenc}
\else % if luatex or xelatex
  \ifxetex
    \usepackage{mathspec}
  \else
    \usepackage{fontspec}
  \fi
  \defaultfontfeatures{Ligatures=TeX,Scale=MatchLowercase}
\fi
% use upquote if available, for straight quotes in verbatim environments
\IfFileExists{upquote.sty}{\usepackage{upquote}}{}
% use microtype if available
\IfFileExists{microtype.sty}{%
\usepackage{microtype}
\UseMicrotypeSet[protrusion]{basicmath} % disable protrusion for tt fonts
}{}
\usepackage[unicode=true]{hyperref}
\hypersetup{
            pdfborder={0 0 0},
            breaklinks=true}
\urlstyle{same}  % don't use monospace font for urls
\usepackage{longtable,booktabs}
\usepackage{graphicx,grffile}
\makeatletter
\def\maxwidth{\ifdim\Gin@nat@width>\linewidth\linewidth\else\Gin@nat@width\fi}
\def\maxheight{\ifdim\Gin@nat@height>\textheight\textheight\else\Gin@nat@height\fi}
\makeatother
% Scale images if necessary, so that they will not overflow the page
% margins by default, and it is still possible to overwrite the defaults
% using explicit options in \includegraphics[width, height, ...]{}
\setkeys{Gin}{width=\maxwidth,height=\maxheight,keepaspectratio}
\IfFileExists{parskip.sty}{%
\usepackage{parskip}
}{% else
\setlength{\parindent}{0pt}
\setlength{\parskip}{6pt plus 2pt minus 1pt}
}
\setlength{\emergencystretch}{3em}  % prevent overfull lines
\providecommand{\tightlist}{%
  \setlength{\itemsep}{0pt}\setlength{\parskip}{0pt}}
\setcounter{secnumdepth}{5}
% Redefines (sub)paragraphs to behave more like sections
\ifx\paragraph\undefined\else
\let\oldparagraph\paragraph
\renewcommand{\paragraph}[1]{\oldparagraph{#1}\mbox{}}
\fi
\ifx\subparagraph\undefined\else
\let\oldsubparagraph\subparagraph
\renewcommand{\subparagraph}[1]{\oldsubparagraph{#1}\mbox{}}
\fi

\author{}
\date{\vspace{-2.5em}}

\begin{document}

{
\setcounter{tocdepth}{1}
\tableofcontents
}
\chapter{Tues Sept 12: Syllabus}\label{tues-sept-12-syllabus}

\section{Instructor Information}\label{instructor-information}

Instructor: Dr.~Amy Hurford\\
Office: Teaching remotely\\
Email: \href{mailto:ahurford@mun.ca}{\nolinkurl{ahurford@mun.ca}}\\
WebEx: \url{https://mun.webex.com/meet/ahurford}\\
Course website: \url{https://ahurford.github.io/math-4190/}

Availability: I will try to reply to emails within 24 hours (excluding
evenings, weekends and holidays). I am always available during the
lecture times. Please email to request a meeting for a different time.
Please check my \href{https://amyhurford.weebly.com/}{schedule} and
suggest a time I am free that works for you.

\section{Course Information}\label{course-information}

TR 2-3.15pm meet on WebEx

Course description:\\
MATH 4190 Mathematical Modelling is intended to develop students' skills
in mathematical modelling and competence in oral and written
presentations. Case studies in modelling will be analyzed. Students will
develop a mathematical model and present it in both oral and report
form.

Course format:\\
Due to online teaching the course will be delivered as a `flipped
classroom'.

Course expectations:\\
Any students that are disruptive, violating university policies, or
acting in a potentially unsafe way will be warned and asked to
leave.\\[2\baselineskip]Learning goals:\\
I consider your completed assignments to be a portfolio of your
knowledge in population and evolutionary ecology. You will also get some
exposure to coding in \texttt{R}. It takes time to become proficient in
a programming language, but the time you will spend coding in this class
will help you towards becoming more proficient. The course content
emphasizes a deeper understanding of fewer concepts. You have the
opportunity to further explore a topic of interest to you for the final
project.

Required Text and Resources:\\
The course materials are online at
\url{https://ahurford.github.io/BIOL-3295-Fall-2020/}. In addition you
will need a computer to install \texttt{R} and \texttt{RStudio}. This
will be covered on Thursday Sept 17 (see Chapter \ref{Rinstall}). Class
announcements and WebEx links will be provided on the course BrightSpace
and your assignments are to be submitted to BrightSpace.

\section{Method of Evaluation}\label{method-of-evaluation}

\begin{itemize}
\tightlist
\item
  27 assignments (equal weighting) - 50\%
\item
  Midterm (due Fri Nov 6 at 5pm) - 15\%
\item
  Final Project (due Monday Dec 14 at 9am) - 35\%
\end{itemize}

You should aim to complete each assignment before the next class, but
assignments will be accepted, without penalty, up to a week later.

Late assignments, labs, and missed midterms, and final exams will be
accommodated as described by University Regulation 6.7.3 and 6.7.5 (see
\url{https://www.mun.ca/regoff/calendar/sectionNo=REGS-0474} for
Regulations).

\section{Additional Policies}\label{additional-policies}

\subsection{Accommodation of students with
disabilities}\label{accommodation-of-students-with-disabilities}

Memorial University of Newfoundland is committed to supporting inclusive
education based on the principles of equity, accessibility and
collaboration. Accommodations are provided within the scope of the
University Policies for the Accommodations for Students with
Disabilities see \url{www.mun.ca/policy/site/policy.php?id=239}.
Students who may need an academic accommodation are asked to initiate
the request with the Glenn Roy Blundon Centre at the earliest
opportunity (see \url{www.mun.ca/blundon} for more information).

\subsection{Academic misconduct}\label{academic-misconduct}

Students are expected to adhere to those principles, which constitute
proper academic conduct. A student has the responsibility to know which
actions, as described under Academic Offences in the University
Regulations, could be construed as dishonest or improper. Students found
guilty of an academic offence may be subject to a number of penalties
commensurate with the offence including reprimand, reduction of grade,
probation, suspension or expulsion from the University. For more
information regarding this policy, students should refer to University
Regulation 6.12.

\subsection{Equity and Diversity}\label{equity-and-diversity}

A safe learning environment will be provided for all students regardless
of race, colour, nationality, ethnic origin, social origin, religious
creed, religion, age, disability, disfigurement, sex (including
pregnancy), sexual orientation, gender identity, gender expression,
marital status, family status, source of income or political opinion.

You should not photograph or record myself, teaching assistants, or
other students in the class without first obtaining permission.
Accommodation will be made for students with special needs.

The sound should be turned off on phones and computers during class.

\section{Additional Supports}\label{additional-supports}

Resources for additional support can be found at:

\begin{itemize}
\item
  \url{www.mun.ca/currentstudents/student/}
\item
  \url{https://munsu.ca/resource-centres/}
\end{itemize}

\section{Tentative course schedule}\label{tentative-course-schedule}

The course schedule is found in the toolbar of the class materials, see
\url{https://ahurford.github.io/BIOL-3295-Fall-2020/}.

The last day to drop the course without academic prejudice is Wednesday
Nov. 4.

\section{Handing in your work}\label{handing-in-your-work}

\subsection{Making figures to hand-in}\label{figures}

The graphs you hand in need to have descriptive axeses and a figure
caption. You may put these elements together using a word processing
software such as \emph{Microsoft Word}. Elements of a good figure
caption:

\begin{itemize}
\item
  Has a label, i.e., ``Figure 1'',
\item
  The first sentences provides a summary of what the figure shows, i.e.,
  ``The price of oranges has increased steadily since 1964'',
\item
  Provide all necessary information to understand everything in the
  figure, i.e., if the figure has no legend, but multiple line
  types/symbols, be sure to indicate what is represented by the
  different symbols. If the axes labels are overly brief due to space
  constraints in the graph, provide a more thorough description in the
  figure caption. If any assumptions have been made in making the
  figure, disclose these, i.e., a point that was excluded from the
  analysis due to being considered an outlier.
\end{itemize}

\subsection{Writing R scripts to hand-in}\label{RScript}

To write your own R scripts follow the guidelines described in Chapter 7
\href{https://ahurford.github.io/quant-guide-all-courses/style.html}{Best
Practices} of \emph{Quantitative training in Biology}. If you are asked
to hand in your R script this means you need to submit an \texttt{.R}
file on Brightspace.

\chapter{Quantitative skills laboratory}\label{lab2}

\begin{center}\rule{0.5\linewidth}{0.5pt}\end{center}

PURPOSE

\begin{enumerate}
\def\labelenumi{\arabic{enumi}.}
\tightlist
\item
  To learn how to record data in electronic format
\item
  To learn how to write hypotheses as equations
\item
  To learn how to choose the appropriate visualizations
\item
  To learn how to make graphs using R Studio
\end{enumerate}

\begin{center}\rule{0.5\linewidth}{0.5pt}\end{center}

Before coming to the laboratory:

\begin{enumerate}
\def\labelenumi{\arabic{enumi}.}
\item
  If you have not already, install
  \href{https://ahurford.github.io/quant-guide-all-courses/install.html}{\texttt{R}
  and \texttt{RStudio}}.
\item
  Before coming to the laboratory read:
\end{enumerate}

\begin{itemize}
\tightlist
\item
  \href{https://ahurford.github.io/quant-guide-all-courses/rintro.html}{Introduction
  to R}
\item
  \href{https://ahurford.github.io/quant-guide-all-courses/graph.html}{Making
  graphs in R}
\item
  \href{https://ahurford.github.io/quant-guide-all-courses/data-entry.html}{Entering
  and loading data}
\end{itemize}

\section*{EXERCISE 1. Data entry and graphing with a continuous
independent
variable}\label{exercise-1.-data-entry-and-graphing-with-a-continuous-independent-variable}
\addcontentsline{toc}{section}{EXERCISE 1. Data entry and graphing with
a continuous independent variable}

\end{document}
